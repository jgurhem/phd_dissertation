\chapter*{Résumé}

Depuis le milieu des années 1990, les bibliothèques de transmission de messages sont les technologies les plus utilisées pour développer des applications parallèles et distribuées.
Des modèles de programmation basés sur des tâches peuvent être utilisés, par exemple, pour éviter les communications collectives sur toutes les ressources comme les réductions, les diffusions ou les rassemblements en les transformant en multiples opérations avec des tâches.
Ensuite, ces opérations peuvent être planifiées par l'ordonnanceur pour placer les données et les calculs de manière à optimiser et réduire les communications de données.

L'objectif principal de cette thèse est d'étudier ce que doit être la programmation basée sur des tâches pour des applications scientifiques et de proposer une spécification de cette programmation distribuée et parallèle, en expérimentant avec plusieurs représentations simplifiées d'applications scientifiques importantes pour TOTAL, et de méthodes linéaire classique dense et creuses.
Au cours de la thèse, plusieurs langages de programmation et paradigmes sont étudiés.
Des méthodes linéaires denses pour résoudre des systèmes linéaires, des séquences de produit matrice vecteur creux et la migration sismique en profondeur pré-empilement de Kirchhoff sont étudiées et implémentées en tant qu'applications basées sur des tâches.
Une taxonomie, basée sur plusieurs de ces langages et paradigmes est proposée.
Des logiciels ont été développés en utilisant ces modèles de programmation pour chaque application simplifiée.
À la suite de ces recherches, une méthodologie pour la programmation de tâches parallèles est proposée, optimisant les mouvements de données en général et, en particulier, pour des applications scientifiques ciblées.

