\chapter{Analysis of Language Properties and Languages Extension}
\label{chap:taxonomy}

This chapter introduces the taxonomy of the languages and the criterion deduced from the usage of the different languages.

\section{Taxonomy}

\subsection{Workflow type}
It represents how the user has to input the dependencies between the tasks.
The user may have to describe a directed acyclic graph (DAG) with the tasks and the associated dependencies.
Another possibility is to use the provided interface of the programming paradigm which runtime system will represent the tasks and dependencies in its way.

\subsection{Granularity}
The granularity of the tasks correspond to the size of the task compared to the whole application.
Fine grained tasks are relatively small in term of code size and execution time.
Coarse grain tasks are bigger.
In the context of parallel and distributed applications, they can be a distributed and parallel sub-part of the whole application.

\subsection{Communications}
In a distributed and parallel applications, there is two types of communications : inside a task and between tasks.
We consider how the communications between tasks are managed.
Does the user has to implement them and the programming paradigm use them to transfer the data between the task ?
Is the user provided with data interface to store its data and has to ask the system to retrieve the needed data ?

\subsection{Scheduling policy}
The scheduling policy is how the tasks are ordered during the execution of the task-based application.
The dependencies between the tasks are respected by the scheduler.
It can also be optimised to reduce the data migrations across the nodes and try to execute tasks on the node where the data are present.

\subsection{Dynamic workflow}
It represents the ability of the scheduler to adapt its policy during runtime depending on the results of the tasks.
For instance, stop the application when the convergence is attained.

\subsection{Type of task}
It represents which code attribute is used as a task.
For instance, it could be a piece of code, a function or a compiled application.

\subsection{Parallel tasks}
Are the tasks of the programming paradigm parallel ?
Is it shared or distributed memory ?

\subsection{Multi-backend}
Is the programming paradigm able to use several backends ?
Is it possible to use code in other languages as tasks ?

\subsection{Control / Data flow graph}
It represents how the programming paradigm interprets the dependencies between the tasks.
It can use the data dependencies, the control dependencies or both of them.

\subsection{Fault tolerant}
Is the programming paradigm fault tolerant ?

\subsection{Installation facilities}
It represents the available facilities to install and use the programming paradigm.
For instance, a Docker file can be provided to install it in a virtual environment or a Docker image can be available on the Docker hub so that it can be used directly.
Moreover, the code can be pre-compiled for several architectures and the binaries provided to the users.

\subsection{Code accessibility}
How can the users access the source code ?
The source code can be provided as a tar ball or in a public repository.

%\subsection{Learnability}
%\subsection{Understandability}
%\subsection{Communicativeness}

\section{Taxonomy Summary}

\begin{table}[H]
	\caption{Taxonomy I}
	\centering
	\newcolumntype{M}[1]{>{\centering\arraybackslash}m{#1}}
	\begin{tabular}{c|c|c|M{3.5cm}|M{3.5cm}|M{2cm}}
		   Paradigm    &   Workflow    & Granularity & Communications                  & Scheduling policy         & Dynamic workflow \\ \hline
		    PaRSEC     &      DAG      & Fine/Coarse & App (MPI)                       & Specialised rules         & Yes              \\
		   YML/XMP     &     Graph     &   Coarse    & App (user defined, file system) & FIFO, changeable          & No               \\
		    DAGMan     &      DAG      &    Fine     & User (file system)              & CONDOR                    & No               \\
		    Swift      &    Worflow    &    Fine     & App (file system)               & Future task               & Yes              \\
		   Swift/T     &    Worflow    & Fine/Coarse & App (MPI)                       & Future task               & Yes              \\
		    Legion     & Tree of tasks &    Fine     & App (GASNet)                    & Differed until it is OK   & Yes              \\
		    OmpSs      &               &   Middle    & User                            & Nanos++                   & No               \\
		     CnC       &   Petri net   &   Middle    & App                             & Dedicated                 & No               \\
		  TensorFlow   &      DAG      &    Fine     & App                             & Dedicated                 & Yes              \\
		Apache Airflow &     Graph     &   Middle    &                                 &                           & No               \\
		     HPX       & DAG(Futures)  &     Any     & App                             & Work-stealing scheduler   & Yes              \\
		   Pegasus     &      DAG      &     Any     & User (file system)              & HTCondor Schedd (Default) & No
	\end{tabular}
\end{table}

\begin{table}[H]
	\caption{Taxonomy II}
	\centering
	\newcolumntype{M}[1]{>{\centering\arraybackslash}m{#1}}
	\begin{tabular}{c|c|c|c|M{3.5cm}|c}
		   Paradigm    &   Type of task   & Parallel tasks & Multi-backend & Control / Data flow graph & Fault tolerant \\ \hline
		    PaRSEC     &     function     &       No       &      No       & Data(PTG) / Control(DTD)  &       No       \\
		   YML/XMP     &    Component     &      Yes       &      Yes      & Both                      &      Yes       \\
		    DAGMan     &   Application    &       No       &      No       & Data                      &                \\
		    Swift      &   Application    &       No       &      Yes      & Data                      &       No       \\
		   Swift/T     & (//) func/script &      Yes       &      Yes      & Data                      &       No       \\
		    Legion     &     Function     &       No       &      No       & Control                   &       No       \\
		    OmpSs      &  Block of code   &       No       &      No       & Control                   &       No       \\
		     CnC       &  Block of code   &       No       &      No       & Both                      &                \\
		  TensorFlow   &     Function     &      Yes       &      No       & Data                      &      Yes       \\
		Apache Airflow & Job on Services  &                &      Yes      & Control                   &                \\
		     HPX       &     function     &       No       &      No       & Data                      &       No       \\
		   Pegasus     &   Application    &      Yes       &      Yes      & Control                   &      Yes
	\end{tabular}
\end{table}

\begin{table}[H]
	\caption{Taxonomy III}
	\centering
	\begin{tabular}{c|c|c}
		Paradigm       & Installation facilities        & Code accessibility         \\ \hline
		PaRSEC         & Complex dependencies + Docker  & Public (bitbucket)         \\
		YML/XMP        & Complex dependencies + VM      & Restrained access (gitlab) \\
		DAGMan         & Binaries                       & Public (github)            \\
		Swift          & Binaries + Building though ant & Public (github)            \\
		Swift/T        & Binaries + Install script      & Public (github)            \\
		Legion         & Simple dependencies + Docker   & Public (github)            \\
		OmpSs          & Docker + VM                    & Public (github)            \\
		CnC            & Binaries + Docker              & Public (github)            \\
		TensorFlow     & Docker + pip + Google Colab    & Public (github)            \\
		Apache Airflow & Docker + pip                   & Public (github)            \\
		HPX            & Binaries + Docker              & Public (github)            \\
		Pegasus        & Binaries + VM                  & Public (github)            \\
	\end{tabular}
\end{table}


\section{Language Properties Relevant for Task-Based Programming}

\section{Languages Extension}