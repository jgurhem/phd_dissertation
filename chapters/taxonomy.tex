\chapter{Taxonomy of Task-Based Programming Models}
\label{chap:taxonomy}

In this chapter, the taxonomy and the properties deduced from the usage of several task based programming models is introduced.
This taxonomy also presents how each property is expressed in the different tasks programming models.

\section{Taxonomy}
The taxonomy of the task based programming models introduced in the previous chapter is detailed in this section.

\subsection{Task Capabilities}
\subsubsection{Architecture}
The architecture supported by the task based programming models is an important property to take into consideration since it will impact the scale at which the task based programming models can be used.
There is two main scale used in high performance computing : the shared memory and the distributed memory.
Therefore, the two values considered for this property are the \textit{shared memory} and the \textit{distributed memory} architectures.

In Table \ref{tab:taxo:prop_architecture}, the architecture which is supported by the considered task based programming models is introduced.

\begin{table}[H]
	\caption{Architecture property for each task based programming model \label{tab:taxo:prop_architecture}}
	\centering
	\begin{multicols}{2}
		\begin{tabular}{cc}
\hline
Programming model & Architecture \\
\hline
Chapel & Distributed Memory\\
Charm++ & Distributed Memory\\
Cilk & Shared Memory\\
ClusterSs & Distributed Memory\\
CnC & Shared Memory\\
Cpp-TaskFlow & Shared Memory\\
HPX & Distributed Memory\\
HTCondor DAGMan & Distributed Memory\\
Kokkos & Shared Memory\\
Legion & Distributed Memory\\
OmpSs & Shared Memory\\
OpenMP & Shared Memory\\
PaRSEC (DTD) & Distributed Memory\\
\hline
\end{tabular}


		\begin{tabular}{cc}
\hline
Programming model & Architecture \\
\hline
PaRSEC (PTG) & Distributed Memory\\
Pegasus & Distributed Memory\\
PyGion & Distributed Memory\\
QUARK & Shared Memory\\
Regent & Distributed Memory\\
StarPU & Distributed Memory\\
Swift & Distributed Memory\\
TBB & Shared Memory\\
TensorFlow & Shared Memory\\
Uintah & Distributed Memory\\
X10 & Distributed Memory\\
YML & Distributed Memory\\
YML+XMP & Distributed Memory\\
\hline
\end{tabular}

	\end{multicols}
\end{table}

\subsubsection{Heterogeneity}
This property indicates if the programming model supports accelerators (for instance, GPUs).
\textit{Explicit} support means that the user has to provide the implementation of the task that will be run on the accelerator.
\textit{Implicit} support means that the tasks can be run on different devices while the user has to provide only one implementation.

In Table \ref{tab:taxo:prop_heterogeneity}, it is shown if the accelerator support is explicit or implicit for the considered task based programming models.

\begin{table}[H]
	\caption{Heterogeneity property for each task based programming model \label{tab:taxo:prop_heterogeneity}}
	\centering
	\begin{multicols}{2}
		\begin{tabular}{cc}
\hline
Programming model & Heterogeneity \\
\hline
Chapel & Implicit\\
Charm++ & Implicit or Explicit\\
Cilk & Explicit\\
ClusterSs & Explicit\\
CnC & Explicit\\
Cpp-TaskFlow & Explicit\\
HPX & Explicit\\
HTCondor DAGMan & Explicit\\
Kokkos & Implicit\\
Legion & Explicit\\
OmpSs & Implicit\\
OpenMP & Implicit\\
PaRSEC (DTD) & Explicit\\
\hline
\end{tabular}


		\begin{tabular}{cc}
\hline
Programming model & Heterogeneity \\
\hline
PaRSEC (PTG) & Explicit\\
Pegasus & Explicit\\
PyGion & Implicit\\
QUARK & Explicit\\
Regent & Implicit\\
StarPU & Explicit\\
Swift & Explicit\\
TBB & Explicit\\
TensorFlow & Implicit\\
Uintah & Implicit\\
X10 & Implicit\\
YML & Explicit\\
YML+XMP & Explicit\\
\hline
\end{tabular}

	\end{multicols}
\end{table}

\subsubsection{Task Granularity}
This property represents the amount of resources on which a task can be executed.
For this study, we consider that either \textit{sequential}, \textit{parallel} or \textit{parallel and distributed} code can be executed as task.
Usually, a sequential task is run as a lightweight thread, a thread or a single process without multi-threading.
A parallel task can also use multiple threads or processes up to one node like OpenMP or MPI on shared memory.
Finally, distributed and parallel resources like multiple nodes from a cluster can be allocated to parallel and distributed tasks.
These tasks could be able to execute MPI or PGAS based code on their allocated resources.

In Table \ref{tab:taxo:prop_task_granularity}, the granularity of the tasks of the considered task based programming models is presented.

\begin{table}[H]
	\caption{Task Granularity property for each task based programming model \label{tab:taxo:prop_task_granularity}}
	\centering
	\begin{multicols}{2}
		\begin{tabular}{cc}
\hline
Programming model & Task Granularity \\
\hline
Chapel & Sequential Task\\
Charm++ & Sequential Task\\
Cilk & Sequential Task\\
ClusterSs & Sequential Task\\
CnC & Sequential Task\\
Cpp-TaskFlow & Sequential Task\\
HPX & Sequential Task\\
HTCondor DAGMan & Parallel Task\\
Kokkos & Sequential Task\\
Legion & Sequential Task\\
OmpSs & Sequential Task\\
OpenMP & Sequential Task\\
PaRSEC (DTD) & Sequential Task\\
\hline
\end{tabular}


		\begin{tabular}{cc}
\hline
Programming model & Task Granularity \\
\hline
PaRSEC (PTG) & Sequential Task\\
Pegasus & Parallel Task\\
PyGion & Sequential Task\\
QUARK & Parallel Task\\
Regent & Sequential Task\\
StarPU & Sequential Task\\
Swift & Parallel Task\\
TBB & Sequential Task\\
TensorFlow & Sequential Task\\
Uintah & Sequential Task\\
X10 & Sequential Task\\
YML & Sequential Task\\
YML+XMP & Parallel Task\\
\hline
\end{tabular}

	\end{multicols}
\end{table}

\subsubsection{Data Handling}
This property describes how the data are accessed in a task.
The data can be accessed \textit{directly}.
For instance, the data can be accessed through function parameters.
The data can also be requested or retrieved from a \textit{container}.
For instance, the data can be retrieved through a future.

In Table \ref{tab:taxo:prop_data_handling}, how to access the data from the tasks is described.
\begin{table}[H]
	\caption{Data Handling property for each task based programming model \label{tab:taxo:prop_data_handling}}
	\centering
	\begin{multicols}{2}
		\begin{tabular}{cc}
\hline
Programming model & Data Handling \\
\hline
Chapel & Direct Access\\
Charm++ & Direct Access\\
Cilk & Direct Access\\
ClusterSs & Direct Access\\
CnC & Direct Access\\
Cpp-TaskFlow & Direct Access\\
HPX & Access through Container\\
HTCondor DAGMan & Direct Access\\
Kokkos & Access through Container\\
Legion & Access through Container\\
OmpSs & Direct Access\\
OpenMP & Direct Access\\
PaRSEC (DTD) & Direct Access\\
\hline
\end{tabular}


		\begin{tabular}{cc}
\hline
Programming model & Data Handling \\
\hline
PaRSEC (PTG) & Direct Access\\
Pegasus & Access through Container\\
PyGion & Direct Access\\
QUARK & Direct Access\\
Regent & Access through Container\\
StarPU & Direct Access\\
Swift & Access through Container\\
TBB & Direct Access\\
TensorFlow & Access through Container\\
Uintah & Direct Access\\
X10 & Direct Access\\
YML & Direct Access\\
YML+XMP & Direct Access\\
\hline
\end{tabular}

	\end{multicols}
\end{table}

\subsubsection{Task Implementation}
This property indicates what kind of interface the user has to fill in to create a task which will be executed by the runtime of the programming model.
The tasks can use a \textit{program} where the user has to provide the parameters through the API.
It can also use a \textit{function pointer} which the user has to pass to the API as well as its parameters.
The programing paradigm can also be based on \textit{pragmas} which are used to delimit and describe the task.
Another possibility is a custom interface like a \textit{function with specific parameters} or a \textit{custom syntax}.

In Table \ref{tab:taxo:prop_task_implementation}, how the tasks are implemented is detailed.
\begin{table}[H]
	\caption{Task Implementation property for each task based programming model \label{tab:taxo:prop_task_implementation}}
	\centering
	\begin{multicols}{2}
		\begin{tabular}{cc}
\hline
Programming model & Task Implementation \\
\hline
Chapel & Dedicated Language\\
Charm++ & Dedicated Language\\
Cilk & Dedicated Language\\
ClusterSs & Dedicated Language\\
CnC & Function Pointer\\
Cpp-TaskFlow & Function Pointer\\
HPX & Function Pointer\\
HTCondor DAGMan & Program\\
Kokkos & Function Pointer\\
Legion & Function with Specific Parameters\\
OmpSs & Pragma\\
OpenMP & Pragma\\
PaRSEC (DTD) & Function Pointer\\
\hline
\end{tabular}


		\begin{tabular}{cc}
\hline
Programming model & Task Implementation \\
\hline
PaRSEC (PTG) & Code Encapsulation\\
Pegasus & Program\\
PyGion & Function Pointer\\
QUARK & Function Pointer\\
Regent & Dedicated Language\\
StarPU & Function Pointer\\
Swift & Function Pointer\\
TBB & Function Pointer\\
TensorFlow & Function with Specific Parameters\\
Uintah & Function Pointer\\
X10 & Dedicated Language\\
YML & Code Encapsulation\\
YML+XMP & Code Encapsulation\\
\hline
\end{tabular}

	\end{multicols}
\end{table}

\subsubsection{Portability Accelerators}
This property indicates what kind of accelerators are supported by the task based programming models.
They can support or generate \textit{CUDA} code and execute it on NVIDIA GPUs.
There is also the possibility of supporting \textit{multiple} accelerator architectures.
This may be achieved by having a suitable backend for each supported accelerator architecture.
Another possibility is to let the \textit{user} chose how and which type of accelerator to use by interfacing with an accelerator programming language.

In Table \ref{tab:taxo:prop_portability_accelerators}, the accelerators supported by the different task based programming models are shown.
\begin{table}[H]
	\caption{Portability Accelerators property for each task based programming model \label{tab:taxo:prop_portability_accelerators}}
	\centering
	\begin{multicols}{2}
		\begin{tabular}{cc}
\hline
Programming model & Portability Accelerators \\
\hline
Chapel & CUDA\\
Charm++ & CUDA\\
Cilk & User Choice\\
ClusterSs & User Choice\\
CnC & User Choice\\
Cpp-TaskFlow & CUDA\\
HPX & CUDA\\
HTCondor DAGMan & User Choice\\
Kokkos & Multiple Accelerators\\
Legion & CUDA\\
OmpSs & Multiple Accelerators\\
OpenMP & Multiple Accelerators\\
PaRSEC (DTD) & CUDA\\
\hline
\end{tabular}


		\begin{tabular}{cc}
\hline
Programming model & Portability Accelerators \\
\hline
PaRSEC (PTG) & CUDA\\
Pegasus & User Choice\\
PyGion & CUDA\\
QUARK & User Choice\\
Regent & CUDA\\
StarPU & CUDA\\
Swift & User Choice\\
TBB & OpenCL\\
TensorFlow & CUDA\\
Uintah & Multiple Accelerator Languages\\
X10 & CUDA\\
YML & User Choice\\
YML+XMP & User Choice\\
\hline
\end{tabular}

	\end{multicols}
\end{table}

\subsection{Task and Data Management}
\subsubsection{Dependency Type}
This property describes how the dependencies between the tasks are provided by the user of the task based programming models.
The user may have to provide \textit{control} dependencies between the tasks.
They describe in which order the tasks can be executed and which tasks can be executed at the time.
\textit{Data} dependencies can also be provided.
This approach is used to infer the dependencies and the parallelism between the tasks by studying how the data are used by the tasks and how they flow from one task to the other.
The user may have to provide \textit{both} dependency type although it is possible to convert a dependency graph type into the other with the appropriate informations.

In Table \ref{tab:taxo:prop_dependency_type}, the dependency type used to describe the dependencies between the tasks in task based programming models is introduced.
\begin{table}[H]
	\caption{Dependency Type property for each task based programming model \label{tab:taxo:prop_dependency_type}}
	\centering
	\begin{multicols}{2}
		\begin{tabular}{cc}
\hline
Programming model & Dependency Type \\
\hline
Chapel & Both\\
Charm++ & Control\\
Cilk & Data\\
ClusterSs & Data\\
CnC & Both\\
Cpp-TaskFlow & Control\\
HPX & Data\\
HTCondor DAGMan & Control\\
Kokkos & Data\\
Legion & Data\\
OmpSs & Both\\
OpenMP & Both\\
PaRSEC (DTD) & Control\\
\hline
\end{tabular}


		\begin{tabular}{cc}
\hline
Programming model & Dependency Type \\
\hline
PaRSEC (PTG) & Data\\
Pegasus & Control\\
PyGion & Data\\
QUARK & Data\\
Regent & Data\\
StarPU & Both\\
Swift & Data\\
TBB & Both\\
TensorFlow & Data\\
Uintah & Data\\
X10 & Data\\
YML & Control\\
YML+XMP & Control\\
\hline
\end{tabular}

	\end{multicols}
\end{table}

\subsubsection{Worker Management}
This property indicates whether the worker thread or process which hosts the tasks in the task based programming models has to be started and maintained by the user (\textit{explicit}) or is provided by the runtime (\textit{implicit}).

In Table \ref{tab:taxo:prop_worker_management}, the worker management is shown for each task based programming model.
\begin{table}[H]
	\caption{Worker Management property for each task based programming model \label{tab:taxo:prop_worker_management}}
	\centering
	\begin{multicols}{2}
		\begin{tabular}{cc}
\hline
Programming model & Worker Management \\
\hline
Chapel & Implicit\\
Charm++ & Implicit\\
Cilk & Implicit\\
ClusterSs & Implicit or Explicit\\
CnC & Implicit\\
Cpp-TaskFlow & Implicit\\
HPX & Implicit or Explicit\\
HTCondor DAGMan & Implicit\\
Kokkos & Implicit\\
Legion & Implicit\\
OmpSs & Implicit\\
OpenMP & Implicit\\
PaRSEC (DTD) & Implicit\\
\hline
\end{tabular}


		\begin{tabular}{cc}
\hline
Programming model & Worker Management \\
\hline
PaRSEC (PTG) & Implicit\\
Pegasus & Implicit\\
PyGion & Implicit\\
QUARK & Implicit\\
Regent & Implicit\\
StarPU & Implicit\\
Swift & Implicit\\
TBB & Implicit\\
TensorFlow & Explicit\\
Uintah & Explicit\\
X10 & Implicit\\
YML & Implicit\\
YML+XMP & Implicit\\
\hline
\end{tabular}

	\end{multicols}
\end{table}

\subsubsection{Interoperability}
This property shows whether task based programming models can be integrated (\textit{callee}) into another language, can call other languages (\textit{caller}) or \textit{both}.

In Table \ref{tab:taxo:prop_interoperability}, the interoperability of each task based programming model is detailed.
\begin{table}[H]
	\caption{Interoperability property for each task based programming model \label{tab:taxo:prop_interoperability}}
	\centering
	\begin{multicols}{2}
		\begin{tabular}{cc}
\hline
Programming model & Interoperability \\
\hline
Chapel & Callee\\
Charm++ & Caller\\
Cilk & Callee\\
ClusterSs & Callee\\
CnC & Callee\\
Cpp-TaskFlow & Callee\\
HPX & Callee\\
HTCondor DAGMan & Caller\\
Kokkos & Callee\\
Legion & Both\\
OmpSs & Callee\\
OpenMP & Callee\\
PaRSEC (DTD) & Callee\\
\hline
\end{tabular}


		\begin{tabular}{cc}
\hline
Programming model & Interoperability \\
\hline
PaRSEC (PTG) & Callee\\
Pegasus & Caller\\
PyGion & Caller\\
QUARK & Both\\
Regent & Both\\
StarPU & Callee\\
Swift & Caller\\
TBB & Callee\\
TensorFlow & Callee\\
Uintah & Caller\\
X10 & Callee\\
YML & Caller\\
YML+XMP & Caller\\
\hline
\end{tabular}

	\end{multicols}
\end{table}

\subsubsection{Data Distribution}
This property describes how the data distribution is handled in the programming model.
\textit{Implicit} data distribution means that the runtime system decides where to place the data on the nodes whereas \textit{explicit} data distribution means that the user has to specify the distribution of the data across the nodes.

In Table \ref{tab:taxo:prop_data_distribution}, the data distribution management in the task based programming models is introduced.
\begin{table}[H]
	\caption{Data Distribution property for each task based programming model \label{tab:taxo:prop_data_distribution}}
	\centering
	\begin{multicols}{2}
		\begin{tabular}{cc}
\hline
Programming model & Data Distribution \\
\hline
Chapel & Implicit\\
Charm++ & Implicit or Explicit\\
Cilk & Not distributed\\
ClusterSs & Implicit\\
CnC & Not distributed\\
Cpp-TaskFlow & Not distributed\\
HPX & Implicit or Explicit\\
HTCondor DAGMan & Implicit\\
Kokkos & Not distributed\\
Legion & Implicit\\
OmpSs & Not distributed\\
OpenMP & Not distributed\\
PaRSEC (DTD) & Explicit\\
\hline
\end{tabular}


		\begin{tabular}{cc}
\hline
Programming model & Data Distribution \\
\hline
PaRSEC (PTG) & Explicit\\
Pegasus & Implicit\\
PyGion & Implicit\\
QUARK & Not distributed\\
Regent & Implicit\\
StarPU & Implicit\\
Swift & Explicit\\
TBB & Not distributed\\
TensorFlow & Not distributed\\
Uintah & Implicit\\
X10 & Implicit\\
YML & Implicit\\
YML+XMP & Implicit\\
\hline
\end{tabular}

	\end{multicols}
\end{table}

\subsubsection{Task Binding}
This property describes how the tasks are bound to the allocated hardware resources.
The binding can be \textit{implicit} when automatically determined by the task based programming runtime or \textit{explicit} when the user has to provide a binding map.

In Table \ref{tab:taxo:prop_task_binding}, the task binding management of the task based programming models is highlighted.
\begin{table}[H]
	\caption{Task Binding property for each task based programming model \label{tab:taxo:prop_task_binding}}
	\centering
	\begin{multicols}{2}
		\begin{tabular}{cc}
\hline
Programming model & Task Binding \\
\hline
Chapel & Implicit\\
Charm++ & Implicit or Explicit\\
Cilk & Implicit\\
ClusterSs & Implicit\\
CnC & Implicit\\
Cpp-TaskFlow & Implicit\\
HPX & Implicit or Explicit\\
HTCondor DAGMan & Implicit\\
Kokkos & Implicit\\
Legion & Implicit\\
OmpSs & Implicit\\
OpenMP & Implicit\\
PaRSEC (DTD) & Implicit or Explicit\\
\hline
\end{tabular}


		\begin{tabular}{cc}
\hline
Programming model & Task Binding \\
\hline
PaRSEC (PTG) & Implicit or Explicit\\
Pegasus & Implicit\\
PyGion & Implicit\\
QUARK & Implicit or Explicit\\
Regent & Implicit\\
StarPU & Implicit\\
Swift & Implicit\\
TBB & Implicit\\
TensorFlow & Implicit\\
Uintah & Implicit\\
X10 & Implicit or Explicit\\
YML & Implicit\\
YML+XMP & Implicit\\
\hline
\end{tabular}

	\end{multicols}
\end{table}

\subsubsection{Task Insertion}
This property indicates if new tasks can be added to the task pool during the execution of the already scheduled tasks.

In Table \ref{tab:taxo:prop_task_insertion}, the insertion of new tasks during the execution of the tasks is described.
\begin{table}[H]
	\caption{Task Insertion property for each task based programming model \label{tab:taxo:prop_task_insertion}}
	\centering
	\begin{multicols}{2}
		\begin{tabular}{cc}
\hline
Programming model & Task Insertion \\
\hline
Chapel & Yes\\
Charm++ & Yes\\
Cilk & Yes\\
ClusterSs & No\\
CnC & No\\
Cpp-TaskFlow & Yes\\
HPX & Yes\\
HTCondor DAGMan & Yes\\
Kokkos & Yes\\
Legion & Yes\\
OmpSs & Yes\\
OpenMP & Yes\\
PaRSEC (DTD) & Yes\\
\hline
\end{tabular}


		\begin{tabular}{cc}
\hline
Programming model & Task Insertion \\
\hline
PaRSEC (PTG) & No\\
Pegasus & No\\
PyGion & Yes\\
QUARK & Yes\\
Regent & No\\
StarPU & No\\
Swift & Yes\\
TBB & Yes\\
TensorFlow & Yes\\
Uintah & No\\
X10 & Yes\\
YML & No\\
YML+XMP & No\\
\hline
\end{tabular}

	\end{multicols}
\end{table}

\subsection{Programming Model Features}
\subsubsection{Dependency Expression}
This property describes how the dependencies between the tasks are represented in the considered model.
The possibilities include a \textit{graph}, a \textit{directed acyclic graph} or DAG, a \textit{tree} and a \textit{Petri Net}.

In Table \ref{tab:taxo:prop_dependency_expression}, how the dependency are expressed in each task based programming model is detailed.
\begin{table}[H]
	\caption{Dependency Expression property for each task based programming model \label{tab:taxo:prop_dependency_expression}}
	\centering
	\begin{multicols}{2}
		\begin{tabular}{cc}
\hline
Programming model & Dependency Expression \\
\hline
Chapel & Directed Acyclic Graph\\
Charm++ & Directed Acyclic Graph\\
Cilk & Directed Acyclic Graph\\
ClusterSs & Directed Acyclic Graph\\
CnC & Petri Network\\
Cpp-TaskFlow & Graph\\
HPX & Directed Acyclic Graph\\
HTCondor DAGMan & Directed Acyclic Graph\\
Kokkos & Directed Acyclic Graph\\
Legion & Tree\\
OmpSs & Graph\\
OpenMP & Graph\\
PaRSEC (DTD) & Directed Acyclic Graph\\
\hline
\end{tabular}


		\begin{tabular}{cc}
\hline
Programming model & Dependency Expression \\
\hline
PaRSEC (PTG) & Directed Acyclic Graph\\
Pegasus & Directed Acyclic Graph\\
PyGion & Tree\\
QUARK & Directed Acyclic Graph\\
Regent & Tree\\
StarPU & Queue\\
Swift & Graph\\
TBB & Graph\\
TensorFlow & Graph\\
Uintah & Directed Acyclic Graph\\
X10 & Directed Acyclic Graph\\
YML & Directed Acyclic Graph\\
YML+XMP & Directed Acyclic Graph\\
\hline
\end{tabular}

	\end{multicols}
\end{table}


\subsubsection{Communication Model}
This property describes how data are sent from a task to another.
The runtime system can use \textit{message passing} (msg), \textit{global address space} (gas) or the \textit{file system} (fs).

In Table \ref{tab:taxo:prop_communication_model}, the communication model used in each task based programming model is shown.
\begin{table}[H]
	\caption{Communication Model property for each task based programming model \label{tab:taxo:prop_communication_model}}
	\centering
	\begin{multicols}{2}
		\begin{tabular}{cc}
\hline
Programming model & Communication Model \\
\hline
Chapel & Global Address Space\\
Charm++ & Global Address Space\\
Cilk & Local Shared Memory\\
ClusterSs & Global Address Space\\
CnC & Local Shared Memory\\
Cpp-TaskFlow & Local Shared Memory\\
HPX & Global Address Space\\
HTCondor DAGMan & File system (I/Os)\\
Kokkos & Local Shared Memory\\
Legion & Global Address Space\\
OmpSs & Local Shared Memory\\
OpenMP & Local Shared Memory\\
PaRSEC (DTD) & Message Passing\\
\hline
\end{tabular}


		\begin{tabular}{cc}
\hline
Programming model & Communication Model \\
\hline
PaRSEC (PTG) & Message Passing\\
Pegasus & File system (I/Os)\\
PyGion & Global Address Space\\
QUARK & Local Shared Memory\\
Regent & Global Address Space\\
StarPU & Message Passing\\
Swift & Message Passing\\
TBB & Local Shared Memory\\
TensorFlow & Local Shared Memory\\
Uintah & Message Passing\\
X10 & Global Address Space\\
YML & File system (I/Os)\\
YML+XMP & File system (I/Os)\\
\hline
\end{tabular}

	\end{multicols}
\end{table}

\subsubsection{Fault Tolerance}
This property indicates if the task based programming models support fault tolerance.

In Table \ref{tab:taxo:prop_fault_tolerance}, the fault tolerance support is given for each task based programming model.
\begin{table}[H]
	\caption{Fault Tolerance property for each task based programming model \label{tab:taxo:prop_fault_tolerance}}
	\centering
	\begin{multicols}{2}
		\begin{tabular}{cc}
\hline
Programming model & Fault Tolerance \\
\hline
Chapel & No\\
Charm++ & Yes\\
Cilk & No\\
ClusterSs & No\\
CnC & No\\
Cpp-TaskFlow & No\\
HPX & No\\
HTCondor DAGMan & Yes\\
Kokkos & No\\
Legion & No\\
OmpSs & No\\
OpenMP & No\\
PaRSEC (DTD) & No\\
\hline
\end{tabular}


		\begin{tabular}{cc}
\hline
Programming model & Fault Tolerance \\
\hline
PaRSEC (PTG) & No\\
Pegasus & Yes\\
PyGion & No\\
QUARK & No\\
Regent & No\\
StarPU & No\\
Swift & No\\
TBB & No\\
TensorFlow & No\\
Uintah & Yes\\
X10 & Yes\\
YML & No\\
YML+XMP & No\\
\hline
\end{tabular}

	\end{multicols}
\end{table}

\subsubsection{Implementation Type}
This property describes how the programming model API is included in an application.
It can be done through a \textit{library}, a \textit{language extension} or a \textit{language}.

In Table \ref{tab:taxo:prop_implementation_type}, the API access is introduced.
\begin{table}[H]
	\caption{Implementation Type property for each task based programming model \label{tab:taxo:prop_implementation_type}}
	\centering
	\begin{multicols}{2}
		\begin{tabular}{cc}
\hline
Programming model & Implementation Type \\
\hline
Chapel & Language\\
Charm++ & Language Extension\\
Cilk & Runtime System\\
ClusterSs & Language Extension\\
CnC & Language\\
Cpp-TaskFlow & Library\\
HPX & Library\\
HTCondor DAGMan & Language\\
Kokkos & Library\\
Legion & Library\\
OmpSs & Language Extension\\
OpenMP & Language Extension\\
PaRSEC (DTD) & Library\\
\hline
\end{tabular}


		\begin{tabular}{cc}
\hline
Programming model & Implementation Type \\
\hline
PaRSEC (PTG) & Library\\
Pegasus & Language\\
PyGion & Language Extension\\
QUARK & Runtime System\\
Regent & Language\\
StarPU & Runtime System\\
Swift & Language\\
TBB & Library\\
TensorFlow & Library\\
Uintah & Language\\
X10 & Language\\
YML & Language\\
YML+XMP & Language\\
\hline
\end{tabular}

	\end{multicols}
\end{table}

\subsubsection{Data Persistence}
This property indicates if the task based programming model supports data persistence.

The support for data persistence is given in Table \ref{tab:taxo:prop_data_persistence}.
\begin{table}[H]
	\caption{Data Persistence property for each task based programming model \label{tab:taxo:prop_data_persistence}}
	\centering
	\begin{multicols}{2}
		\begin{tabular}{cc}
\hline
Programming model & Data Persistence \\
\hline
Chapel & No\\
Charm++ & Yes\\
Cilk & No\\
ClusterSs & No\\
CnC & No\\
Cpp-TaskFlow & No\\
HPX & No\\
HTCondor DAGMan & Yes\\
Kokkos & No\\
Legion & No\\
OmpSs & No\\
OpenMP & No\\
PaRSEC (DTD) & No\\
\hline
\end{tabular}


		\begin{tabular}{cc}
\hline
Programming model & Data Persistence \\
\hline
PaRSEC (PTG) & No\\
Pegasus & Yes\\
PyGion & No\\
QUARK & No\\
Regent & No\\
StarPU & No\\
Swift & Limited\\
TBB & No\\
TensorFlow & No\\
Uintah & No\\
X10 & Yes\\
YML & Yes\\
YML+XMP & Yes\\
\hline
\end{tabular}

	\end{multicols}
\end{table}

\subsubsection{Scheduler Location}
This property describes where the scheduler instances of the task based programming models are located.
It can be \textit{centralized} where there only one scheduler instance that manages all the tasks.
The tasks can also be managed at the local level in the workers in a \textit{distributed} way.

In Table \ref{tab:taxo:prop_scheduler_location}, the location of the scheduler is shown for each task based programming model.
\begin{table}[H]
	\caption{Scheduler Location property for each task based programming model \label{tab:taxo:prop_scheduler_location}}
	\centering
	\begin{multicols}{2}
		\begin{tabular}{cc}
\hline
Programming model & Scheduler Location \\
\hline
Chapel & Centralized\\
Charm++ & Distributed\\
Cilk & Centralized\\
ClusterSs & Centralized\\
CnC & Centralized\\
Cpp-TaskFlow & Centralized\\
HPX & Distributed\\
HTCondor DAGMan & Centralized\\
Kokkos & Centralized\\
Legion & Distributed\\
OmpSs & Centralized\\
OpenMP & Centralized\\
PaRSEC (DTD) & Distributed\\
\hline
\end{tabular}


		\begin{tabular}{cc}
\hline
Programming model & Scheduler Location \\
\hline
PaRSEC (PTG) & Distributed\\
Pegasus & External\\
PyGion & Distributed\\
QUARK & Centralized\\
Regent & Distributed\\
StarPU & Distributed\\
Swift & Centralized\\
TBB & Centralized\\
TensorFlow & Centralized\\
Uintah & Distributed\\
X10 & Distributed\\
YML & Centralized\\
YML+XMP & Centralized\\
\hline
\end{tabular}

	\end{multicols}
\end{table}

%\subsubsection{}
%\begin{table}[H]
%	\caption{}
%	\centering
%	\input{"chapters/taxonomy/_table_taxo_.tex"}
%\end{table}


%\subsection{Other ideas}
%\subsubsection{Scheduling policy}
%The scheduling policy is how the tasks are ordered during the execution of the task-based application.
%The dependencies between the tasks are respected by the scheduler.
%It can also be optimized to reduce the data migrations across the nodes and try to execute tasks on the node where the data are present.
%
%\subsubsection{Dynamic workflow}
%It represents the ability of the scheduler to adapt its policy during runtime depending on the results of the tasks.
%For instance, stop the application when the convergence is attained.
%
%
%\subsubsection{Multi-backend}
%Is the programming paradigm able to use several backends ?
%Is it possible to use code in other languages as tasks ?


%\subsection{Learnability}
%\subsection{Understandability}
%\subsection{Communicativeness}

\section{Taxonomy Summary}

\begin{table}[H]
	\caption{Taxonomy I}
	\centering
	\newcolumntype{M}[1]{>{\centering\arraybackslash}m{#1}}
	\begin{tabular*}{\textwidth}{m{0.125\textwidth}m{0.125\textwidth}m{0.125\textwidth}m{0.125\textwidth}m{0.125\textwidth}m{0.125\textwidth}m{0.125\textwidth}}
\hline
 & \rotatebox{90}{Dependency Type} & \rotatebox{90}{Task Insertion} & \rotatebox{90}{Data Distribution} & \rotatebox{90}{Worker management} & \rotatebox{90}{Task Binding} & \rotatebox{90}{Interoperability}\\ \hline
Chapel & b & yes & i & i & i & callee\\
Charm++ & c & yes & i/e & i & i/e & caller\\
Cilk & d & yes & n & i & i & callee\\
ClusterSs & d & no & i & i/e & i & callee\\
CnC & b & no & n & i & i & callee\\
Cpp-TaskFlow & c & yes & n & i & i & callee\\
HPX & d & yes & i/e & i/e & i/e & callee\\
HTCondor DAGMan & c & yes & i & i & i & caller\\
Kokkos & d & yes & n & i & i & callee\\
Legion & d & yes & i & i & i & both\\
OmpSs & b & yes & n & i & i & callee\\
OpenMP & b & yes & n & i & i & callee\\
PaRSEC (DTD) & c & yes & e & i & i/e & callee\\
PaRSEC (PTG) & d & no & e & i & i/e & callee\\
Pegasus & c & no & i & i & i & caller\\
PyGion & d & yes & i & i & i & caller\\
QUARK & d & yes & n & i & i/e & both\\
Regent & d & no & i & i & i & both\\
StarPU & b & no & i & i & i & callee\\
Swift & d & yes & e & i & i & caller\\
TBB & b & yes & n & i & i & callee\\
TensorFlow & d & yes & n & e & i & callee\\
Uintah & d & no & i & e & i & caller\\
X10 & d & yes & i & i & i/e & callee\\
YML & c & no & i & i & i & caller\\
YML+XMP & c & no & i & i & i & caller\\
\hline
\end{tabular*}

\end{table}

\begin{table}[H]
	\caption{Taxonomy II}
	\centering
	\newcolumntype{M}[1]{>{\centering\arraybackslash}m{#1}}
	\begin{tabular*}{\textwidth}{m{0.1111111111111111\textwidth}m{0.1111111111111111\textwidth}m{0.1111111111111111\textwidth}m{0.1111111111111111\textwidth}m{0.1111111111111111\textwidth}m{0.1111111111111111\textwidth}m{0.1111111111111111\textwidth}m{0.1111111111111111\textwidth}}
\hline
 & \rotatebox{90}{Task Granularity} & \rotatebox{90}{Nested Tasks} & \rotatebox{90}{Task Implementation} & \rotatebox{90}{Heterogeneity} & \rotatebox{90}{Architecture} & \rotatebox{90}{Data Handling} & \rotatebox{90}{Portability Accelerators}\\ \hline
Chapel & s & yes & s & i & d & d & CUDA\\
Charm++ & s & yes & s & i/e & d & d & CUDA\\
Cilk & s & yes & s & e & s & d & user\\
ClusterSs & s & no & s & e & d & d & user\\
CnC & s & no & fp & e & s & d & user\\
Cpp-TaskFlow & s & yes & fp & e & s & d & CUDA\\
HPX & s & yes & fp & e & d & c & CUDA\\
HTCondor DAGMan & p & no & pgm & e & d & d & user\\
Kokkos & s & yes & fp & i & s & c & s\\
Legion & s & yes & sf & e & d & c & CUDA\\
OmpSs & s & yes & p & i & s & d & s\\
OpenMP & s & yes & p & i & s & d & s\\
PaRSEC (DTD) & s & no & fp & e & d & d & CUDA\\
PaRSEC (PTG) & s & no & s & e & d & d & CUDA\\
Pegasus & p & no & pgm & e & d & c & user\\
PyGion & s & yes & fp & i & d & d & CUDA\\
QUARK & p & no & fp & e & s & d & user\\
Regent & s & yes & s & i & d & c & CUDA\\
StarPU & s & no & fp & e & d & d & CUDA\\
Swift & p & no & fp & e & d & c & user\\
TBB & s & yes & fp & e & s & d & OpenCL\\
TensorFlow & s & no & sf & i & s & c & CUDA\\
Uintah & s & no & fp & i & d & d & m\\
X10 & s & yes & s & i & d & d & CUDA\\
YML & s & no & s & e & d & d & user\\
YML+XMP & p & no & s & e & d & d & user\\
\hline
\end{tabular*}

\end{table}

\begin{table}[H]
	\caption{Taxonomy III}
	\centering
	\begin{tabular*}{\textwidth}{m{0.125\textwidth}m{0.125\textwidth}m{0.125\textwidth}m{0.125\textwidth}m{0.125\textwidth}m{0.125\textwidth}m{0.125\textwidth}}
\hline
 & \rotatebox{90}{Dependency Expression} & \rotatebox{90}{Communication Model} & \rotatebox{90}{Fault Tolerance} & \rotatebox{90}{Implementation Type} & \rotatebox{90}{Data persistence} & \rotatebox{90}{Scheduler Location}\\ \hline
Chapel & DAG & gas & no & lang & no & c\\
Charm++ & SDAG & gas & yes & ext & yes & d\\
Cilk & DAG & sm & no & rt & no & c\\
ClusterSs & DAG & gas & no & ext & no & c\\
CnC & Petri & sm & no & lang & no & c\\
Cpp-TaskFlow & Graph & sm & no & lib & no & c\\
HPX & DAG & gas & no & lib & no & d\\
HTCondor DAGMan & DAG & fs & yes & lang & yes & c\\
Kokkos & DAG & sm & no & lib & no & c\\
Legion & Tree & gas & no & lib & no & d\\
OmpSs & Graph & sm & no & ext & no & c\\
OpenMP & Graph & sm & no & ext & no & c\\
PaRSEC (DTD) & DAG & msg & no & lib & no & d\\
PaRSEC (PTG) & DAG & msg & no & lib & no & d\\
Pegasus & DAG & fs & yes & lang & yes & e\\
PyGion & Tree & gas & no & ext & no & d\\
QUARK & DAG & sm & no & rt & no & c\\
Regent & Tree & gas & no & lang & no & d\\
StarPU & Queue & msg & no & rt & no & d\\
Swift & Graph & msg & no & lang & limited & c\\
TBB & Graph & sm & no & lib & no & c\\
TensorFlow & Graph & sm & no & lib & no & c\\
Uintah & DAG & msg & yes & lang & no & d\\
X10 & DAG & gas & yes & lang & yes & d\\
YML & DAG & fs & no & lang & yes & c\\
YML+XMP & DAG & fs & no & lang & yes & c\\
\hline
\end{tabular*}

\end{table}


\section{Language Properties Relevant for Task-Based Programming}

\section{Languages Extension}